%==================================
% (c) Jan Tulak, 2017
%%%%%%%%%%%%%%%%%%%%%%%%%%%%%%%%%%%%%%%%%%%%%%%%%%%%%%%%%%%%%%%%%%%%%%%%%%%%%

\chapter{Introduction}\label{chap:introduction}
%----------------------------------------------------------------------


In software projects with long life, even an initially clean codebase can
become messy and complicated. Moreso when we speak about open-source
projects where the original creators left yers ago and new people of
various capabilities and knowledge continue the development.

In such projects, new functionality is added to the existing code with
minimal changes to the rest of the project. This may simplify the merging
of these changes, as any responsible person can easily understand what the
change does. But on the other side, in the long term, it turns the code into
a disordered chaos.

The result is increasingly more difficult to maintain and test, and as a single functionality can be spread over many portions of the project, any change requires more and more attention and time.

xfsprogs, a package of tools for XFS filesystem, is such a project. While
the filesystem itself is subject to careful scrutiny from the Linux kernel
community, the tools like {\tt mkfs.xfs}\footnote{Formats a
partition as XFS.}, {\tt fsck.xfs}\footnote{Usually checks and repairs errors in an
existing filesystem. But for XFS it only tells the user what other
tools to use.} and others are not so publicly exposed and get a
lot less attention. From our experience with working on this project, it
happens that only one or two persons other than the author of a patch may read
it, and miss some subtle side effect the change has. Sometimes, the large
set of tests XFS maintains captures this bug, sometimes it does not and it
is noticed much later.

On this point, it is important to highlight that despite the naming
convention, each {\tt mkfs} tool is completely independent project and, for
example, {\tt mkfs.xfs} and {\tt mkfs.ext4} do not share any code except
system libraries.

Some parts of this code are more than 20 years old (see chapter 1 for a
detailed history of XFS) and in need of intensive cleaning. The test
suite (project xfstests) maintains hundreds of more or less complex tests,
but these are limited in what they can detect as they usually work in this
way: make a filesystem, then test that, so many errors in {\tt mkfs.xfs}
are difficult to capture or notice. XFS also uses an automatic static
analysis from Coverity, which is useful, but the project has no good data
on the reliability of this analysis.

With the approval of David Chinner, then the maintainer of XFS, we began the
refactoring of mkfs.xfs, which was overdue. The goal of this work is to
repay the technical debt accumulated over the years. That means not only
fixing some long-known issues and cleaning particularly complex parts of
the code, but also making structural changes to minimize the amount of code
that must be added or changed during the regular development (adding and
removing features of the filesystem). These changes should slow the build-up
of the technical debt in the future.

After implementing these changes, this work should verify how effective
the currently used tests and analysis are. Even if some testing and
analysis methods can be used only on a part of the code, the results, when
compared with other tools, still provides an estimate about the soundess
and completeness of every used method.

At the same time, this work can also be seen as a review of how well
various analysis and verification methods perform on real and in-production
code.

The refactoring was done in two parts. One set of changes was merged into
upstream in June 2016 (xfsprogs 4.7\footnote{The releasing of xfsprogs is
		tightly coupled with releases of XFS kernel module, which
		is part of Linux. Thus, xfsprogs uses the same version
		number that the respective XFS kernel module and Linux has.}),
the other set is, at the time of writing, still in development. The
versions of xfsprogs at different stages of this work were:
\begin{itemize}
\item Before the beginning of refactoring -- xfsprogs 4.6.
\item After merging the first part -- xfsprogs 4.7.
\item Before merging the second part -- xfsprogs 4.11 at the time of writing.
\item After applying the second part -- not yet merged, changes only in a local repository.
\end{itemize}

This work is structured as follows: First, information about XFS and
mkfs.xfs are provided in the first chapter. In the following, third chapter, we look
at the refactoring done and discuss the changes. After that, another
chapter is dedicated to explaining formal analysis and verification,
describing common techniques, and we point out notable tools, from which
we select few to use in the fifth and sixth chapter, where the testing
environment is described and results analysed.

