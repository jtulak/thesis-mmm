%==================================
% (c) Jan Tulak, 2017
%%%%%%%%%%%%%%%%%%%%%%%%%%%%%%%%%%%%%%%%%%%%%%%%%%%%%%%%%%%%%%%%%%%%%%%%%%%%%

\chapter{Used Techniques and procedures}\label{chap:techniques}
%----------------------------------------------------------------------

In this chapter, we discuss which techniques and models of formal analysis and verification are useful for the code of {\tt mkfs.xfs}. Let us at first define important constraints that are limiting or directing our choice.

First of all, we are analysing a single-threaded application. This greatly reduces the state space and means that we also can use methods that do not allow for concurrency. On the other side, given that the program accepts user input, some variables has an infinite number of potential values and any method based on state space checks has to cope with this fact.

With respect to possible difficulties with implementing various advanced
methods and their required proficiency, it was decided to begin with well-known
and used tools and gradually move from these production-ready, easy to use
solutions to tools requiring more of user input.

Some tools will be used multiple times, because while, for example, {\em
	Coverity} can be used on the whole source of {\em xfsprogs}, other
	tools like {\em CPAChecker} require modelling of the environment and so
	for practical reasons, they are run only on an extracted part of {\tt
		mkfs\_xfs.c}, where things like access to disk devices are
		removed to create a maximally self-contained code. This means
		that we are not be able to detect errors in some parts of the
		code, but it is nearly impossible to model and test a whole
		operating system and we have to make the cut somewhere.

Thus, to be able to compare the various tools directly, some of the tools are run both on the full code as well as on the dissected part.

A potentially interesting comparison could be if there exist some tools using neural networks and deep learning as an integral part of their algorithms. For example as a heuristic to drive the selection of inference rules in theorem proving, or for spotting error patterns~\footnote{Some attempts in modelling a code are hinted in {\em On the Naturalness of Software} by Abram Hindle: \url{http://dl.acm.org/citation.cfm?id=2902362}.} However, these approaches are even more complex and experimental and would probably deserve a standalone work on their own.
% TODO http://dl.acm.org/citation.cfm?id=2902362 download and read... search for deep learning sources

List of tools that were used in this work (in no particular order): Coverity,
CPAChecker, CppCheck and the analysis in GCC and Clang compilers.


%======================================================================
\section{Testing Environment}\label{chap:techniques:env}
%----------------------------------------------------------------------

The tools were run in a Docker container\footnote{Simply stated, a
	container is an image that has been started, similar to the
		difference between a running virtual machine and its
		on-disk virtual HDD image. Unlike virtualization,
		containers are only processes isolated from the rest of the
system using kernel capabilities, like {\em cgroups} and {\em chroot}.
Docker is a specific implementation~\cite{docker} of containers.} based on Fedora Linux
25. The use of containers ensures a clean and identical environment for
every tool and every run. Images with CPAChecker and CppCheck are published
in Docker Hub and all the recipes to build and use them are provided with
this work. Image with Coverity could not be published, because it contains
confident information\footnote{Jan Ťulák had an access to Red Hat Coverity license
server as a Red Hat employee. However, the server information and some tools
	Red Hat provided with Coverity are considered confidential.}.


Every image has its own starting script for easier manipulation -- {\tt
run.sh}.  This script creates a container from the image and mounts the
directory with xfsprogs (or any other directory which is passed to it).
Then, if necessary, it can pass few options to the script started in
the container.

{\tt run-test.sh} is the script started by Docker after creating the container.
This script copies xfsprogs from the mounted directory to another one, so it
does not change the original repository in any way. In this copied directory
the script then starts whatever tool it is prepared for.

Users can, if they wish to do so, enter an interactive shell in the container
instead of starting the tool. Also, it is possible to skip the copying or to
run {\tt make clean}. For an automated run descripte in
\Cref{chap:techniques:processing} neither of this is necessary, but these
options are useful for manual experiments.


Coverity, GCC and Clang are run using {\tt csbuild}, a tool to plug static
analyzers into the build process~\cite{csbuildMan}. Because csbuild attempts to
use all supported analyzers it finds, the images for each tool we are testing
are modifed to contain only the single specific tool we need, but no other.
%_____________________________________________________
\subsection{CPAChecker}
%.....................................................
Docker image: {\tt jtulak/cpacheck}~\cite{dockerCPAChecker}

CPAChecker can not be run on a full source code, but requires
preprocessing~\cite{cpacheckerGettingStarted}, which has to be partially
remade for every tested revision (depending on what and how much changed),
       only selected commits were tested.
%_____________________________________________________
\subsection{CppCheck}
%.....................................................
Docker image: {\tt jtulak/cppcheck}~\cite{dockerCPPCheck}

CppCheck is also used in {\em Codacy}, an automated code review application
with Github integration. Results from Codacy are included for comparison.

Because CppCheck does not need preprocessed code, it was reasonable to use
it for every commit in my changes.

When running this tool, default configuration was used, and all types of
messages were enabled. No custom rules were used and the invocation of
CppCheck on whole {\tt xfsprogs/mkfs/} directory was:

{\tt cppcheck --enable=all mkfs/}

\TODO{Find some non-default rules for cppcheck and test with them? Or try to
	write a rule for issues found by other tools.}


%_____________________________________________________
\subsection{Coverity}
%.....................................................
Coverity was used both manually in a Docker container, and automatically,
using the public Coverity service for open source projects, which is part
of standard xfsprogs development process, to compare the results between
those two instances.

%_____________________________________________________
\subsection{GCC}
%.....................................................
The {\em GNU project C and C++ compiler} is used for compiling xfsprogs and it
has some static analysis capabilities, because it has to understand the code to
compile it. The only difference in its use from standard configuration is to
use the most strict reporting.

%_____________________________________________________
\subsection{Clang}
%.....................................................
Clang is another C/C++ compiler. It is not used by xfsprogs, but can compile
the code as well, so we can compare it with GCC and other analysis tools. To use
clang instead of GCC, we decided that the easiest way is to rename clang binary
to gcc and let xfsprogs behave as if it was gcc, rather than modify autotools
configuration.

%======================================================================
\section{Results Processing}\label{chap:techniques:processing}
%----------------------------------------------------------------------
The outputs of these tools has different syntax and verbosity, but we had to
find a way, how to compare them, both between the tools and across revisions,
     despite some of the tools finding a great amount of issues. A set of
     scripts to help both with automating the tests and with analysing was
     created.

First, there is tool {\tt parse.py}, which can automatically run all the tools
across specified revisions. It takes care of changing the revisions, starting
every docker container again and finally, it organises the outputs in a logical
way: in a specified directory, it creates a subdirectory for every revision
(using the revision's short hash as the directory name) and each such directory
then contains log files with outputs from each tool.

The output files are not modified in the first step, but to simplify their
parsing, it is useful to preprocess some of those files (namely from GCC and
							 Clang) with script
{\tt format-outputs.sh}, to remove color formatting escape sequences and
unnecessary compiler outputs.  Such data may be useful for some further
analysis, but for the next step, it would only make the parsing more complex.

In the last step, script {\tt parse.py}, when supplied with the output
directory, translates the different syntaxes into a single inner
representation, which can be then used to simply compute deltas between
different revisions.

The algorithms to complete these deltas has one known issue: if there are
multiple issues with the same message (e.g. because variable with name {\tt
				       foo} was declared, but not used, in
				       multiple functions) and later some of
these issues are fixed, the number of issues is correct, but the indicated
lines may be incorrect. This is because the script has to cope with changing
code; an issue on line X in one revision can be on line Y in another one, and
that would require employing much more complex algorithms that would use
information from git and understand which lines moved where. Thus, in such
case, the behaviour selecting specific instancies of the same kind of issue is
undefined.


The classification of issues into one of the style, error or security category
is only indicative. For GCC and Clang it is based on what specific flag for the
compiler enables reporting of any specific issue. We attempted to estimate what
consequences can these issues have in general, whether they can cause a bigger
problem, or if their greatest danger is in being misleading or not a good
practice.

From the issues found in whole xfsprogs, we configured only one compiler flag
to be detected as something else than style issue: {\tt -Wfloat-equal}.
Directly comparing two floats is not reliable due to their representation
in memory~\cite{floatsComparing}. Thus, this kind of issue is certainly
something that can work differently than expected and deserves more
attention, than, say, shadowing a global variable with local one.

This kind of issue was not found in mkfs itself and in xfsprogs, it appears
only on two places, both to set the precision of a printf. In other words,
the only occurrences of direct comparison of floats are intentional testing
for imprecise numbers and can be considered false positives in the context
of the program. Clang finds this issue as well. In comparison, Coverity
does not raise it and it seems that it can somehow infer the intention of
the use.

For CppCheck, the classification uses the category reported by CppCheck itself:
If CppCheck reports an issue as anything else than style issue, it is
considered a potential error, but it did not report anything else than style
issue. Coverity reports also a categorization. In xfsprogs, the only two
categories that were reported are quality and security.

Finally, while it is possible to find differences between revisions within the
results of a single tool, albeit with the small instability in case of multiple
similar entries mentioned above, doing this between tools on a single revision
proved a much more complex task. Every tool describes the same issue with
different words, so to be able to automatically compute any differences, such
an algorithm would have to understand the issue in all details. Thus,
   cross-tool differences are not computed automatically, but manually for the
   cases where it is reasonable given to the amount of issues that has to be
   analysed.
