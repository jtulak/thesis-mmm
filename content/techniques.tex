%==================================
% (c) Jan Tulak, 2017
%%%%%%%%%%%%%%%%%%%%%%%%%%%%%%%%%%%%%%%%%%%%%%%%%%%%%%%%%%%%%%%%%%%%%%%%%%%%%

\chapter{Used Techniques and procedures}\label{chap:techniques}
%----------------------------------------------------------------------

\TODO{Rewrite the introduction of this chapter once the tests are done.}
In this chapter, we discuss which techniques and models of formal analysis and verification are useful for the code of {\tt mkfs.xfs}. Let us at first define important constraints that are limiting or directing our choice.

First of all, we are analysing a single-threaded application. This greatly reduces the state space and means that we also can use methods that do not allow for concurrency. On the other side, given that the program accepts user input, some variables has an infinite number of potential values and any method based on state space checks has to cope with this fact.

With respect to possible difficulties with implementing various advanced
methods and their required proficiency, it was decided to begin with well-known
and used tools and gradually move from these production-ready, easy to use
solutions to tools requiring more of user input.

Some tools will be used multiple times, because while, for example, {\em
	Coverity Scan} can be used on the whole source of {\em xfsprogs}, other
	tools like {\em CPAChecker} require modelling of the environment and so
	for practical reasons, they will be run only on an extracted part of
{\tt mkfs\_xfs.c}, where things like access to disk devices are removed to
create a maximally self-contained code. This means that we will not be able to
detect errors in some parts of the code, but it is nearly impossible to model
and test a whole operating system and we have to make a cut somewhere. 

Thus, to be able to compare the various tools directly, some of the tools are run both on the full code as well as on the dissected part.

A potentially interesting comparison could be if there exist some tools using neural networks and deep learning as an integral part of their algorithms. For example as a heuristic to drive the selection of inference rules in theorem proving, or for spotting error patterns~\footnote{Some attempts in modelling a code are hinted in {\em On the Naturalness of Software} by Abram Hindle: \url{http://dl.acm.org/citation.cfm?id=2902362}.}
% TODO http://dl.acm.org/citation.cfm?id=2902362 download and read... search for deep learning sources

List of tools intend to be used (in no particular order): Coverity,
	CPAChecker, some Lint-like tools, CppCheck and analysis in GCC itself.


%======================================================================
\section{Testing Environment}\label{chap:techniques:env}
%----------------------------------------------------------------------

The tools were run in a Docker container\footnote{Simply stated, a
	container is an image that has been started, similar to the
		difference between a running virtual machine and its
		on-disk virtual HDD image. Unlike virtualization,
		containers are only processes isolated from the rest of the
system using Kernel capabilities, like {\em cgroups} or {\em chroot}.
Docker is a specific implementation~\cite{docker}} based on Fedora Linux
25. The use of containers ensures a clean and identical environment for
every tool and every run. Images with CPAChecker and CppCheck are published
in Docker Hub and all the recipes to build and use them are provided with
this work. Image with Coverity could not be published, because it contains
confident information\footnote{We had an access to Red Hat Coverity license
server as a Red Hat employee. However, the server information and some tools
	Red Hat provided with Coverity are considered confidential. }.

%_____________________________________________________
\subsection{CPAChecker}
%.....................................................
Docker image: {\tt jtulak/cpacheck}~\cite{dockerCPAChecker}

CPAChecker can not be run on a full source code, but requires
preprocessing~\cite{cpacheckerGettingStarted}, which has to be partially
remade for every tested revision (depending on what and how much changed),
       only selected commits were tested.
%_____________________________________________________
\subsection{CppCheck}
%.....................................................
Docker image: {\tt jtulak/cppcheck}~\cite{dockerCPPCheck}

CppCheck is also used in {\em Codacy}, an automated code review application
with Github integration. Results from Codacy are included for comparison.

Because CppCheck does not need preprocessed code, it was reasonable to use
it for every commit in my changes.

When running this tool, default configuration was used, and all types of
messages were enabled. No custom rules were used and the invocation of
CppCheck on whole {\tt xfsprogs/mkfs/} directory was:

{\tt cppcheck --enable=all mkfs/}

\TODO{Find some non-default rules for cppcheck and test with them? Or if
	some other tool finds anything, try to write a rule for that
		issue.}


%_____________________________________________________
\subsection{Coverity}
%.....................................................
Coverity was used both manually in a Docker container, and automatically,
using the public Coverity service for open source projects, which is part
of standard xfsprogs development process, to compare the results between
those two instances.

%_____________________________________________________
\subsection{GCC}
%.....................................................
The {\em GNU project C and C++ compiler} is used for compiling xfsprogs and
it has some static analysis capabilities. The only difference in its use
from standard configuration is to use the most strict reporting.

