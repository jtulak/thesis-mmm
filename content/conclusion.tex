%==================================
% (c) Jan Tulak, 2017
%%%%%%%%%%%%%%%%%%%%%%%%%%%%%%%%%%%%%%%%%%%%%%%%%%%%%%%%%%%%%%%%%%%%%%%%%%%%%

\chapter{Conclusion} \label{chap:conclusion}
%----------------------------------------------------------------------

As we have found, the accumulation of technical debt in long-living
projects can impair the understanding of the code and pick on as a
snowball, as each change requires more ad-hoc adjustments and edits than
the previous one. When this happens, it is important to devote some effort
to clean the code, even if it can take a long time, because otherwise, the
situation will only grow worse.

As \Cref{chap:refactoring} shows, we have begun this work and successfully
merged the first set of changes. Because the development process continues
slower than we expected, not all of the desired changes were finished
merged before this work was published. This does not change our plan in
merging them. Rather, we only have to find better processes that will limit
long delays and speed up the merging of the changes into the project. Some
of these possible changes were discussed in
\Cref{chap:refactoring:summary}.

We also compared the online Coverity service xfsprogs is using with few
other tools, to see how effective the current reviews and tests are. The
results in the \Cref{chap:results} speaks rather well. All the found issues
in mkfs.xfs were only of a minor importance, even though different tools
found different kinds of issues. With higher sensitivity, the tools were
reporting more of those issues, but with diminishing importance and growing
amount of false positives.

While it might be useful to incorporate the other tools and let the tools
use more aggressive assumptions, it is uncertain if the return of
investment into fixing a large amount of minor and stylistic issues is
positive, or if the effort is better spent on keeping the XFS filesystem
competitive with other modern and much younger filesystems.
