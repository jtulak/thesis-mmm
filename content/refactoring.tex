%==================================
% (c) Jan Tulak, 2017
%%%%%%%%%%%%%%%%%%%%%%%%%%%%%%%%%%%%%%%%%%%%%%%%%%%%%%%%%%%%%%%%%%%%%%%%%%%%%

\chapter{Refactoring of mkfs.xfs} \label{chap:refactoring}
%----------------------------------------------------------------------

The idea of changes described in this work is to rewrite a complex and
chaotic code for parsing user input with a table that holds values like
minimum/maximum, default values, conflicts and others. Thus, instead of
ad-hoc conditions and operations, there will be just one global structure,
well documented and easily readable and extendable. This structure should
hold also the user-entered values and limit code and variables duplication
as much as possible.

During the development, we had to repeatedly solve conflicts with changes
from other developers that got merged into xfsprogs while we were still
working on my changes. That led me to cut the work into multiple parts.
That way, others could benefit from changes that were already done and we
would not have to maintain this so many patches up to date. There are two
main patchsets which are accompanied by few small and enclosed changes that
could be easily submitted independently.

Naming convention used in this work is the same as what is internally used
in xfsprog.
\begin{description}
\item[Option] Can be referred as a section. The highest-level argument of
	mkfs, starting with a dash.  E.g. {\tt -b} or {\tt -d }. Most
	options have a mandatory argument consisting of suboptions.
\item[Suboption] Can be also referred as some section's option. Consists
	of one or more items in a format {\tt name=value} separated by a
	comma, but no space. The value can be optional, e.g. when the
	suboption is a boolean flag.
\end{description}

Look on the example in the \Cref{lst:refactoring:example}. The command has
two options and two suboptions. The options are {\tt -f}, which does not
have any argument and serves as a {\em force} flag, so mkfs does overwrite
any existing filesystem on the target device.

The second option is {\tt -d}, which has arguments for setting up
non-default values for data section.  There are two used suboptions of this
option: {\tt file} is only a flag, that tells mkfs that the target device
is not a block device, but a regular file (and thus mkfs should not use
direct IO, or compute blocksize differently). {\tt size} has its own
argument and denotes the size of the data section, 10 GB in this case.
Because nothing else is specified, the size of other sections is computed
automatically. The full list of options is in \Cref{lst:xfs:synopsis} in
\Cref{chap:xfs:mkfs}.

\begin{lstlisting}[frame=none, basicstyle=\footnotesize\ttfamily,
language=Bash, numbers=none, numberstyle=\tiny\color{black},caption=
{An example of mkfs.xfs invocation.},
label={lst:refactoring:example}]
mkfs.xfs -f -d file,size=10G /foo/bar
\end{lstlisting}

%======================================================================
\section{Development processes}\label{chap:refactoring:processes}
%----------------------------------------------------------------------

At first we will briefly describe the development processes and tools used for
xfsprogs, which are similar as tools and processes used for Linux Kernel
development.

Most of the communication is happening on a mailing
list\footnote{Specifically linux-xfs@vger.kernel.org.} while IRC chat is
used for some less
important and more day to day issues. The code is hosted in a Git repository,
but only selected maintainers have a write access.

Any commit an author wants to get merged into the code has to be submitted
as a patch to the mailing list. There the patch awaits a review -- that is,
some other developer has to check the changes and append his or her
signature to this patch. Once the patch is reviewed and if there are no
objections, the maintainer will merge it in a batch with other changes once
in a time (for xfsprogs, it is usually about twice a month).

However, there are many unwritten rules and customs, that are not apparent
at first and a new developer finds about them usually only when she or he
breaks such a rule.

An example of such an unwritten rule is the exact coding style and the use of
a code style checking script {\tt checkpatch.pl} originating in Kernel
community and is part of Linux Kernel source. Such a rules has their
meaning and helps to keep a consistent style throughout xfsprogs, but the
fact that they are not documented causes unnecessary issues and delays.

%======================================================================
\section{Initial codebase}\label{chap:refactoring:initialcodebase}
%----------------------------------------------------------------------

Almost all the important code we were changing is located in {\tt
mkfs/xfs\_mkfs.c} file. The code before the first patchset was merged can
be accessed in the project's Git repository as a version 4.6. Git revision
hash for this version is {\tt 09033e35}. In this revision, the parsing of
user input works as follows.

In {\tt main(int argc, char **argv)} function is a loop using a standard
{\tt getopt} from {\tt unistd.h} to detect an option like {\tt -d} for data
section or {\tt -l} for log section. For options that has some arguments, a
nested loop uses custom functions to parse specific suboptions and their
values.

As an example, here is the beginning of aforementioned loops, as it is in
the code, and some issues with this code.

\begin{lstlisting}[frame=none, basicstyle=\footnotesize\ttfamily,
language=C, numbers=none, numberstyle=\tiny\color{black},
caption= {Part of option-parsing loop from mkfs.xfs with additional
comments.},
label={lst:refactoring:loopexample}]
while ((c = getopt(argc, argv, "b:d:i:l:L:m:n:KNp:qr:s:CfV")) != EOF) {
	switch (c) {
	case 'C':
	case 'f':
		force_overwrite = 1;
		break;
	case 'b':
		p = optarg;
		/*
		 * This nested loop will parse the argument of -b, which is
		 * a list of suboptions separated by a comma, but not space.
		 */
		while (*p != '\0') {
			char	*value;
			
			/*
			 * The getsubopt() function removes the first suboption
			 * from the 'p' variable and returns a number
			 * representing the specific suboption, while
			 * saving its value (if any) to 'value'.
			 */
			switch (getsubopt(&p, (constpp)bopts, &value)) {

			/*
			 * an example of how one suboption is parsed.
			 */
			case B_LOG:
				if (!value || *value == '\0')
					reqval('b', bopts, B_LOG);
				if (blflag)
					respec('b', bopts, B_LOG);
				if (bsflag)
					conflict('b', bopts, B_SIZE,
							B_LOG);
				blocklog = atoi(value);
				if (blocklog <= 0)
					illegal(value, "b log");
				blocksize = 1 << blocklog;
				blflag = 1;
				break;
\end{lstlisting}

While the {\tt -f} option is simple, in case of {\tt -b
log=XX}\footnote{Here and on other places, the block option is used as an
example, because this option has only two suboptions, so it can be shown in
full if needed.} we can see how the parsing can get complex. The code tests
if the value is not empty and if it is, it raises an error. Then it tests
whether this specific option was already used, because repeated
specification of the same option is prohibited\footnote{Respecification is
	forbidden for this reason: consider, what happens if a user uses
	this combination of options: {\tt  -b size=4k -d size=1000b -b
	size=512}, where the {\tt b} suffix in a number denotes a block. At
	first, blocksize is set to one value, a size of data section is
	computed based on this value and then the blocksize is changed.
	Thus, any following use of blocksize will have a value different
	than what was used for the first computation. This could be
	countered by computing all values after all options are parsed, yet
	it would still be ambiguous and might behave differently than the
	user expected. Forbidding it is a cleaner and safer approach.} Mkfs
	allows to specify the block size both in explicit size in bytes and
	in a logarithmic scale, but only one of these options can be used
	at a time. So the code also has to check if the other variant was
	used and compute both values.

Some options has a test directly within this assignment for conflicting
options, others simply set up the value and test the conflicts later, after
the {\tt getopt} loop. Yet another options uses both of these ways, depending on what the author of
each change considered a better solution, how it was required to achieve a given
functionality or how this part of code was changed over time.

You can see that most of the things happening in this part of code are
rather generic -- all options has to check for the respecification, whether
a required value is present, or possibly whether the value is in a certain
range of valid values.

However, nothing of these universal tasks is automated -- every single
option has to reimplement the same tests. Some options has almost all logic
in it's {\tt case} where it is at leas in one place. But options with more
complex dependencies and conflicts have only part of it's logic there and
the rest of it is in a section of code following the main loop, in ad-hoc
tests and computations.

To further complicate it, some parts of {\tt mkfs.xfs} are more than 20
years old and the coding style and the general approach to specific things
changed since then, but the old code did not. If such old code needs a
change, there is always a risk that the editing programmer assumes a
different behaviour similar to the one that newer options have, but that
assumption is incorrect.

Also the six variables specific for {\tt -b log} are not explicitly tied
together and because almost every option has a similar mix of multiple
variables, it is difficult to keep all the important ones in a mental image
of the code and always use the correct one. Many of these variables are
unnecessary or reduntant, so in some cases, the values are copied from one
to another and if a change is put into a wrong place, a specific condition
may cause that the changed value is overwritten later on with the old one,
etc.

It is easy to see what could go bad in this: When changing one option, it was
possible to forget to change the other one. If a test was done after {\tt
getopt}, any other option that would modify a value\footnote{See {\tt -i
size=x} and {\tt -i log=y}. In the code, both options modify the same variables
and differs only in accepted values. {\tt size} expects a number of bytes while
{\tt log} expects a base two logarithm value.}, which is used for a computation
in another option, could overwrite the value and cause a conflict without a
notice.

Any new option required a careful reading through the existing code and
possibly place some checks on multiple places. Thus it was difficult to know
when any value is checked and safe for use in further computations.


The consequence of these issues is that {\tt mkfs.xfs} did a bad job of
validating user input from the command line and an unexpected value could
cause that the worst case; the created filesystem could fail after some use
and lose user's data. Even if an issue was detected and the specific
error fixed, the minimal code reuse meant that other options could still be
susceptible to the same or similar issue.

%======================================================================
\section{First patchset}\label{chap:refactoring:first}
%----------------------------------------------------------------------

As is shown \Cref{chap:refactoring:initialcodebase}, the situation was not ideal and the
state of the code led to many known issues. David Chinner, then maintainer
of XFS, presented a set of patches as an RFC\footnote{Request for comment -
	signalling, that the presented patches are not meant to be merged,
but the author wants to hear other people's thoughts about these changes.}
in November 2013~\cite{davidsPatches} in an attempt to raise a discussion.
However, nobody joined him and David Chinner himself did not continue in
pressing this matter for few years. Here is an excerpt from his RFC:

\begin{displayquote}
This is still a work in progress, but is complete enough to get
feedback on the general structure. The problem being solved here is
that mkfs does a terrible job of input validation from the command
line, has huge amounts of repeated code in the sub options
processing loops and has many, many unnecessary variable for
tracking simply things like whether a parameter was specified.

This patchset introduces a parameter table structure that is used to
define the parameters and their constraints. Things like minimum and
maximum valid values, default values, conflicting options, etc are
all contained within the table, so all the "policy" is found in a
single place.

\ldots

The flow on effect of this is that we can remove the many, many
individual variables and start passing the option structures to
functions rather than avoiding using functions because passing so
many variables is messy and nasty. IOWs, it lays the groundwork for
factoring xfs\_mkfs.c into something more than a bunch of spagetti...
\end{displayquote}

When we joined the XFS team and began with the refactoring in
2015~\cite{myFirstPatches}, we picked up this patchset and brought it up to
date with the codebase that in some parts changed substantially in the two
years. Once the patches were applicable for the current code, we began
fixing functional issues and adding further changes.


This lasted until May 2016, when this patchset was merged into the upstream
repository~\cite{finalPatchset1,finalPatchset1Announce}.
These changes implemented the core parts from the desired state. The
implementation of the basic table made the {\tt mkfs\_xfs.c} file more
readable, even if it was possible to remove only basic checks. It also
brought a much more strict input validation, so few of the existing tests
in xfstests had to be updated and a new test was created, with the goal to
watch only for input validation, whether {\tt mkfs.xfs} correctly accepts
or refuses any given combination of options and values.

Size and commits of this patchset are described in
\Cref{lst:refactoring:initialcodebase}. It is 19 patches that are
grouped by the initial author, in this case David Chinner and Jan Ťulák.

\begin{lstlisting}[frame=none, basicstyle=\footnotesize\ttfamily,
language=Bash, numbers=none, numberstyle=\tiny\color{black},caption= {Git
statistics for the first patchset~\cite{finalPatchset1}. Note: Git
attributes changes only to the first author of each commit.},
label={lst:refactoring:initialcodebase}]
Dave Chinner (15):
  xfsprogs: use common code for multi-disk detection
  mkfs: sanitise ftype parameter values.
  mkfs: Sanitise the superblock feature macros
  mkfs: validate all input values
  mkfs: factor boolean option parsing
  mkfs: validate logarithmic parameters sanely
  mkfs: structify input parameter passing
  mkfs: getbool is redundant
  mkfs: use getnum_checked for all ranged parameters
  mkfs: add respecification detection to generic parsing
  mkfs: table based parsing for converted parameters
  mkfs: merge getnum
  mkfs: encode conflicts into parsing table
  mkfs: add string options to generic parsing
  mkfs: don't treat files as though they are block devices

Jan Tulak (4):
  mkfs: move spinodes crc check
  mkfs: unit conversions are case insensitive
  mkfs: add optional 'reason' for illegal_option
  mkfs: conflicting values with disabled crc should fail

 include/Makefile        |    5 +-
 include/xfs_multidisk.h |   73 ++
 libxfs/init.c           |    6 +
 libxfs/linux.c          |   11 +-
 man/man8/mkfs.xfs.8     |   45 +-
 mkfs/Makefile           |    2 +-
 mkfs/maxtrres.c         |    2 +-
 mkfs/proto.c            |   58 +-
 mkfs/xfs_mkfs.c         | 1983 +++++++++++++++++++++++++++++------------------
 mkfs/xfs_mkfs.h         |   89 ---
 repair/xfs_repair.c     |   44 +-
 11 files changed, 1417 insertions(+), 901 deletions(-)
 create mode 100644 include/xfs_multidisk.h
 delete mode 100644 mkfs/xfs_mkfs.h
\end{lstlisting}

%_____________________________________________________
\subsection{Timeline and progress}
%.....................................................

\begin{itemize}
	\item {\em November 2013} -- Dave Chinner submits his RFC.
	\item {\em May 2015} -- We are beginning the work on this patchset.
	\item {\em June 2015} -- The first published version. It contains
		only minor changes except updating and fixing the most
		serious errors. We are getting the first feedback.
	\item {\em March 2016} -- Another version submitted, this time with
		more custom changes.
	\item {\em April 2016} -- Further big changes. Some patches are
		reverted to older versions, while a new patch is added.
	\item {\em May 2016} -- Changes are made only in specific patches,
		no new version of the whole set is submitted.
	\item {\em June 2016} -- The patchset is accepted and merged into
		the repository.
\end{itemize}

%_____________________________________________________
\subsection{Description of important changes}
%.....................................................

The key part of this patchset is the creation of {\tt opt\_params} table.
It is a structure that is holds all the important values for a specific
option in one place, easily acessible and consistent across whole file.

\begin{lstlisting}[frame=none, basicstyle=\footnotesize\ttfamily,
language=C, numbers=none, numberstyle=\tiny\color{black},
caption= {Definition of the table.}]
struct opt_params {
	const char	name;
	const char	*subopts[MAX_SUBOPTS];

	struct subopt_param {
		int		index;
		bool		seen;
		bool		str_seen;
		bool		convert;
		bool		is_power_2;
		int		conflicts[MAX_CONFLICTS];
		long long	minval;
		long long	maxval;
		long long	defaultval;
	}		subopt_params[MAX_SUBOPTS];
};
\end{lstlisting}

The meaning of the specific fields is this:
\begin{description}
\item[name] {\em MANDATORY}
  Name is a single char, e.g., for '-d file', name is 'd'.

\item[subopts] {\em MANDATORY}
  Subopts is a list of strings naming suboptions. In the example above,
  it would contain "file". The last entry of this list has to be NULL.

\item[subopt\_params] {\em MANDATORY}
  This is a list of structs tied with subopts. For each entry in subopts,
  a corresponding entry has to be defined.
\end{description}


The {\tt subopt\_param} has the following members:
\begin{description}
\item[index] {\em MANDATORY}
    This number, starting from zero, denotes which item in subopt\_params
    it is. The index has to be the same as is the order in subopts list,
    so we can access the right item both in subopt\_params and subopts.

\item[seen] {\em INTERNAL}
    Do not set this flag when defining a subopt. It is used to remeber that
    this subopt was already seen, for example for conflicts detection.

\item[str\_seen] {\em INTERNAL}
    Do not set. It is used internally for respecification, when some options
    has to be parsed twice - at first as a string, then later as a number.

\item[convert] {\em OPTIONAL}
    A flag signalling whether the user-given value can use suffixes.
    If you want to allow the use of user-friendly values like 13k, 42G,
    set it to true.

\item[is\_power\_2] {\em OPTIONAL}
    An optional flag for subopts where the given value has to be a power
    of two.

\item[conflicts] {\em MANDATORY}
    If your subopt is in a conflict with some other option, specify it.
    Accepts the .index values of the conflicting subopts and the last
    member of this list has to be LAST\_CONFLICT.

\item[minval, maxval] {\em OPTIONAL}
    These options are used for automatic range check and they have to be
    always used together in pair. If you do not want to limit the max value,
    use something like UINT\_MAX. If no value is given, then you must either
    supply your own validation, or refuse any value in the 'case
    X\_SOMETHING' block. If you forget to define the min and max value, but
    call a standard function for validating user's value, it will cause an
    error message notifying you about this issue.

    (Said in another way, you can not have minval and maxval both equal
    to zero. But if one value is different: minval=0 and maxval=1,
    then it is OK.)

\item[defaultval] {\em MANDATORY}
    The value used if user specifies the subopt, but no value.
    If the subopt accepts some values (-d file=[1|0]), then this
    sets what is used with simple specifying the subopt (-d file).
    A special SUBOPT\_NEEDS\_VAL can be used to require a user-given
    value in any case.
		
\end{description}

{\tt opt\_params} is instantiated for every option category, e.g.
\Cref{lst:refactoring:instantiation} shows instantiation for {\tt
-b}.

\begin{lstlisting}[frame=none, basicstyle=\footnotesize\ttfamily,
language=C, numbers=none, numberstyle=\tiny\color{black},
caption= {Instantiation of the table for block options.},
label={lst:refactoring:instantiation}]
struct opt_params bopts = {
	.name = 'b',
	.subopts = {
#define	B_LOG		0
		"log",
#define	B_SIZE		1
		"size",
		NULL
	},
	.subopt_params = {
		{ .index = B_LOG,
		  .conflicts = { B_SIZE,
				 LAST_CONFLICT },
		  .minval = XFS_MIN_BLOCKSIZE_LOG,
		  .maxval = XFS_MAX_BLOCKSIZE_LOG,
		  .defaultval = SUBOPT_NEEDS_VAL,
		},
		{ .index = B_SIZE,
		  .convert = true,
		  .is_power_2 = true,
		  .conflicts = { B_LOG,
				 LAST_CONFLICT },
		  .minval = XFS_MIN_BLOCKSIZE,
		  .maxval = XFS_MAX_BLOCKSIZE,
		  .defaultval = SUBOPT_NEEDS_VAL,
		},
	},
};
\end{lstlisting}

With this structure, many functions had to be completely rewritten or
added, but the result was that the option parsing loop could be greatly
simplified. For comparison, here is the nested loop from
\ref{lst:refactoring:loopexample} code example after this patchset was
applied. You can see that the section for {\tt B\_LOG} is now much cleaner
(no branching, only few assignments) and the generic logic was moved away
into a function shared with other options as can be seen in
\Cref{lst:refactoring:parsingAfterFirst}.

\begin{lstlisting}[frame=none, basicstyle=\footnotesize\ttfamily,
language=C, numbers=none, numberstyle=\tiny\color{black},
caption= {Part of option-parsing loop from mkfs.xfs after the first patch
set.},
label={lst:refactoring:parsingAfterFirst}]
case 'b':
	p = optarg;
	while (*p != '\0') {
		char	**subopts = (char **)bopts.subopts;
		char	*value;

		switch (getsubopt(&p, (constpp)subopts,
					&value)) {
		case B_LOG:
			blocklog = getnum(value, &bopts, B_LOG);
			blocksize = 1 << blocklog;
			blflag = 1;
			break;
\end{lstlisting}

Another important issue fixed in this set was the behaviour difference when
mkfs.xfs is run to create a filesystem on a block device and in a file on
another filesystem.

The issue was that if the target was a file, but {\tt -d file} was not
specified, mkfs behaved as if the target is a block device. That however
meant, that if the underlying block device had e.g. sector size 512B, on
that a filesystem with sector size 4kB was created, then mkfs used the
(incorrect) 512B size. On top of that, mkfs also used direct IO bypassing
caches in the operating system and when on an SSD, called {\tt discard}
ioctls, erasing changes that the user made to the file before
creating the filesystem and possibly other data too.

This was mitigated by automatic detection of whether the target is a
regular file or a block device, and by changing the flow of the program on
various places where the difference between file and device was important.

However, there were still many issues that were not addressed. The
conflicting options were only enumerated, without any additional
information, and thus the field was usable only for always conflicting
options, like {\tt -b log|size} -- it did not help with conditional conflicts.
For example, checksums for metadata, enabled with {\tt -m crc}, works only
on newer version of metadata format: {\tt -m crc -i attr=1} is conflicting,
but {\tt -i attr=2} is not. Such tests still had to be done as before.
Also, it was possible to specify conflicts only between suboptions of a
single option.


%======================================================================
\section{Second patchset}\label{chap:refactoring:second}
%----------------------------------------------------------------------

Once this change was merged and provided a stable point so we did not have to
keep so much code in my own local repository up to date with upstream, we
began to work on the second set of changes. We submitted an
RFC of these
changes in December 2016~\cite{secondSetRFC}. Such a big and complex
changes are something that most of the developers postpone, so it is
usually reviewed only by the maintainer when nobody else starts it. In this
case, however, the maintainer changed in late December -- Eric Sandeen took
this position instead of David Chinner.

Size of this patchset in the first RFC is 22 patches and the patches can be
seen in \Cref{lst:refactoring:secondSet}.
\begin{lstlisting}[frame=none, basicstyle=\footnotesize\ttfamily,
language=Bash, numbers=none, numberstyle=\tiny\color{black},caption= {Git
statistics for the second patchset~\cite{secondSetRFC}.},
label={lst:refactoring:secondSet}]
Jan Tulak (22):
  mkfs: remove intermediate getstr followed by getnum
  mkfs: merge tables for opts parsing into one table
  mkfs: extend opt_params with a value field
  mkfs: change conflicts array into a table capable of cross-option
    addressing
  mkfs: add a check for conflicting values
  mkfs: add cross-section conflict checks
  mkfs: Move opts related #define to one place
  mkfs: move conflicts into the table
  mkfs: change conflict checks to utilize the new conflict structure
  mkfs: change when to mark an option as seen
  mkfs: add test_default_value into conflict struct
  mkfs: expand conflicts declarations to named declaration
  mkfs: remove zeroed items from conflicts declaration
  mkfs: rename defaultval to flagval in opts
  mkfs: replace SUBOPT_NEEDS_VAL for a flag
  mkfs: Change all value fields in opt structures into unions
  mkfs: use old variables as pointers to the new opts struct values
  mkfs: prevent sector/blocksize to be specified as a number of blocks
  mkfs: subopt flags should be saved as bool
  mkfs: move uuid empty string test to getstr()
  mkfs: remove duplicit checks
  mkfs: prevent multiple specifications of a single option

 mkfs/xfs_mkfs.c | 2952 +++++++++++++++++++++++++++++++++++--------------------
 1 file changed, 1864 insertions(+), 1088 deletions(-)

\end{lstlisting}

Together, these two issues caused that despite my urging, there was not much
reaction until March. In March we submitted another version, this time
intentionally not as an RFC. We also mentioned to few people that this is
part of my thesis.

The review of the second set revealed many disputable points and it become
certain that these patches will need further changes. The second part of
my changes is focused mostly on conflicts detection and allows for almost
all checks to be removed from the code as ad-hoc solutions, as the new
structures and functions take care of them automatically. Any programmer
making a change just has to correctly specify values in a {\tt struct
opt\_params}, write in a list of conflicting options, and the validation of
the new option is guaranteed to work correctly and seamlessly.

To make the process faster, we decided to split this patchset into multiple
smaller ones, which can get fixed, reviewed and merged faster. The first
group focused on extending the options table not only for encoding validity
range and basic conflicts, but also for user input.

Most notable changes are functions {\tt parse|get|set\_conf\_val}
-- a set of functions to manipulate the user input values. This is a key
difference from the RFC. There the values were manipulated as pointers to the
table, but other developers raised objections. Most notably Dave Chinner, who
rebutted my worries about the use of setters and getters in a reply to my
e-mail where we suggested them as another option~\cite{unifyTypes1, unifyTypes2}

Compare code examples \ref{lst:refactoring:aliasing} and
\ref{lst:refactoring:getter}.  The first example with aliases keeps a lot of
seemingly unconnected variables in the code where the programmer does not know
where exactly the pointer leads to. And even if he finds the first assignment,
it is possible to mistakenly override the target address. In the second
example, with setters and getters, it is apparent on the first glance, where
the value is to be written or read. And it is impossible to mistakenly alter
the target address. The disadvantage of using setters is that it is no longer possible to do in-situ increments/decrements (e.g. {\tt i++;}), however this is only a minor issue.

\begin{lstlisting}[frame=none, basicstyle=\footnotesize\ttfamily,
language=C, numbers=none, numberstyle=\tiny\color{black},
label={lst:refactoring:aliasing},
caption={Pointer aliases in RFC of the second set.}]
long long *agcount = &opts[OPT_D].subopt_params[D_AGCOUNT].value;

// ... lines skipped
*agcount = foo(bar);

// ... lines skipped
if (*agcount < SOME_CONSTANT)
	// do something
\end{lstlisting}

\begin{lstlisting}[frame=none, basicstyle=\footnotesize\ttfamily,
language=C, numbers=none, numberstyle=\tiny\color{black},
label={lst:refactoring:getter},
caption={Setters and getters in later version of the second set.}]
set_conf_val(OPT_D, D_AGCOUNT, foo(bar));

// ... lines skipped
if (get_conf_val(OPT_D, D_AGCOUNT) < SOME_CONSTANT)
	// do something
\end{lstlisting}

Furthermore, this approach allows for a verification of all values saved into
the table for the whole run of the program. However, after attempting to
implement this feature, we found out this is not possible to add at this moment.
While we know valid bounds for user input values, some of these values are then
recomputed and can get out of these bounds, while still being valid. An example
of this is {\tt L\_SUNIT}\footnote{{\tt L\_SUNIT} specifies the alignment of
log writes.}, which is specified as a number of 512-byte blocks. However, it is
later multiplied by the 512, at which moment it can get out of the valid bounds
specified for input.

Proposed but not yet implemented solution is to utilize the existing
infrastructure and create a new pseudo option for the table, which would not be
visible for the end user, but would hold all the internal variables for which
bound range (or any other condition, like being power of 2) can be applied.
This topic was briefly discussed in replies to one of the patches in this
set~\cite{secondSetSplitOtherVars}, because one of the other developers, Luis
R. Rodriguez, has a work in progress that requires such infrastructure to be
implemented.

Because even this smaller set was not fully accepted even the end of April,
and adressing the issues other developers raised required too much time, we
furtherfocused on the next part of
this work and did not used the second patchset in formal analysis.

Size of the smaller set can be seen in
\Cref{lst:refactoring:firstSmallerGit}.
\begin{lstlisting}[frame=none, basicstyle=\footnotesize\ttfamily,
language=Bash, numbers=none, numberstyle=\tiny\color{black},caption= {Git
statistics for the first part of the second set after its breaking into
smaller parts~\cite{secondSetSplitFirst}.},
label={lst:refactoring:firstSmallerGit}]
Jan Tulak (12):
  mkfs: Save raw user input field to the opts struct
  mkfs: rename defaultval to flagval in opts
  mkfs: remove intermediate getstr followed by getnum
  mkfs: merge tables for opts parsing into one table
  mkfs: extend opt_params with a value field
  mkfs: create get/set functions for opts table
  mkfs: save user input values into opts
  mkfs: replace variables with opts table: -b,d,s options
  mkfs: replace variables with opts table: -i options
  mkfs: replace variables with opts table: -l options
  mkfs: replace variables with opts table: -n options
  mkfs: replace variables with opts table: -r options

 mkfs/xfs_mkfs.c | 2457 ++++++++++++++++++++++++++++++++-----------------------
 1 file changed, 1420 insertions(+), 1037 deletions(-)
\end{lstlisting}


%_____________________________________________________
\subsection{Timeline and progress}
%.....................................................

\begin{itemize}
	\item {\em June 2016} -- The first patchset is accepted and merged into
		the repository, beginning of the work on the second set.
	\item {\em August 2016} -- First draft of changes of the second set~\cite{secondSetPreRFC}.
	\item {\em December 2016} -- RFC of the full second set. This gained just a little attention.
	\item {\em March 2017} -- Second set without the RFC. This version attracted enough attention to be useful, and provided valuable feedback.
	\item {\em March 2017} -- Vault conference in Boston. We met some other developers personally and debated some of the changes. This helped to raise some attention towards my patches.
	\item {\em April 2017} -- resubmitting only part of the second set with requested changes and setters/getters as an addition. This generated a lot of feedback.
\end{itemize}


%======================================================================
\section{Summary}\label{chap:refactoring:summary}
%----------------------------------------------------------------------

Part of the changes was successfully merged into the project in time.
However, some other patches gained the necessary attention too late and all
the found issues could not be fixed or changed before the deadline for this
work. These changes will get merged eventually.

The difficulties were analysed in an attempt to avoid these delays in the future.
Our hypothesis is that the set as a whole was too big and complex, which effect
was multiplied by unintentionally not adhering to unwritten standards. A
proposed process change for further iterations is to send few smaller patches
more often and wait with other changes depending on those submitted until they
are accepted.

The higher activity on the last patchset, which was just a subset of the second
big set, seems to confirm this hypothesis, however, more iterations is needed.

