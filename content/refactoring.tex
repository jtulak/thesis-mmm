\chapter{Refactoring of mkfs.xfs} \label{chap:refactoring}

The idea of changes described in this work is to rewrite a complex and
chaotic code with a table that holds values like minimum/maximum, default
values, conflicts and others. Thus, instead of ad-hoc conditions and
operations, there will be just one global structure, well documented and
easily readable and extendable. This structure should hold also the
user-entered values and limit code and variables duplication as much as
possible.

\section{Development processes}\label{chap:refactoring:processes}

At first I describe the development processes and tools used for xfsprogs,
which are similar as tools and processes used for Linux Kernel development.

Most of the communication is happening on a mailing
list\footnote{Specifically linux-xfs@vger.kernel.org.} while IRC chat is
used for some less
important and more day to day issues. The code is hosted in a GIT repository,
but only selected maintainers have a write access.

Any commit an author wants to get merged into the code has to be submitted
as a patch to the mailing list. There the patch awaits a review -- that is,
some other developer has to check the changes and append his or her
signature to this patch. Once the patch is reviewed and if there are no
objections, the maintainer will merge it in a batch with other changes once
in a time (for xfsprogs, it is usually about twice a month).

However, there are many unwritten rules and customs, that are not apparent
at first and a new developer finds about them usually only when she or he
breaks such a rule.

An example of such an unwritten rule is the exact coding style and the use of
a code style checking script {\tt checkpatch.pl} originating in Kernel
community and is part of Linux Kernel source. Such a rules has their
meaning and helps to keep a consistent style throughout xfsprogs, but the
fact that they are not documented causes unnecessary issues and delays.

\section{First patchset}\label{chap:refactoring:first}
As is shown in~\ref{chap:xfs:mkfs}, the situation was not ideal and the
state of the code led to many known issues. David Chinner, then maintainer
of XFS, presented a set of patches as an RFC\footnote{Request for comment -
	signalling, that the presented patches are not meant to be merged,
but the author wants to hear other people's thoughts about these changes.}
in November 2013~\cite{davidsPatches} in an attempt to raise a discussion,
however, nobody joined it and David Chinner himself didn't continue in
pressing this matter for few years.

When I joined the XFS team and began with the refactoring in
2015~\cite{myFirstPatches}, I picked up this patchset and brought it up to
date with the codebase that in some parts changed substantially in the two
years. Once the patches were applicable for the current code, I began
fixing functional issues and adding further changes.

During the development, I had to repeatedly solve conflicts with changes
from other developers that got merged into xfsprogs. That led me to cut the
work into multiple parts. That way, others could benefit from changes that
were already done and I wouldn't have to maintain this so many patches up
to date.

The first part took until May 2016, when it was merged into the upstream
repository~\cite{finalPatchset1,finalPatchset1Announce}.
The first patchset implemented the core parts from the desired state. The
implementation of the basic table made the {\tt mkfs\_xfs.c} file more
readable, even if it was possible to remove only basic checks. It also
brought a much more strict input validation, so few of the existing tests
in xfstests had to be updated and a new test was created, with the goal to
watch only for input validation, whether {\tt mkfs.xfs} correctly accepts
or refuses any given combination of options and values.

Size of this patchset is 19 patches and the total changes are:
\begin{lstlisting}[frame=none, basicstyle=\footnotesize\ttfamily, language=Bash, numbers=none, numberstyle=\tiny\color{black},caption= {Git statistics for the first patchset~\cite{finalPatchset1}.}]
include/Makefile        |    5 +-
 include/xfs_multidisk.h |   73 ++
 libxfs/init.c           |    6 +
 libxfs/linux.c          |   11 +-
 man/man8/mkfs.xfs.8     |   45 +-
 mkfs/Makefile           |    2 +-
 mkfs/maxtrres.c         |    2 +-
 mkfs/proto.c            |   58 +-
 mkfs/xfs_mkfs.c         | 1983 +++++++++++++++++++++++++++++------------------
 mkfs/xfs_mkfs.h         |   89 ---
 repair/xfs_repair.c     |   44 +-
11 files changed, 1417 insertions(+), 901 deletions(-)
\end{lstlisting}


\section{Second patchset}\label{chap:refactoring:second}

Once this change was merged and provided a stable point so I didn't have to
keep so much code in my own local repository up to date with upstream, I
began to work on the second set of changes. I submitted an
RFC of these
changes in December 2016~\cite{secondSetRFC}. Such a big and complex
changes are something that most of the developers postpone, so it is
usually reviewed only by the maintainer when nobody else starts it. In this
case, however, the maintainer changed in late December -- Eric Sandeen took
this position instead of David Chinner.

Together, these two issues caused that despite my urging, there wasn't much
reaction until March, when I submitted another version, this time
intentionally not as an RFC. I also mentioned to few people that this is
part of my thesis.

The review of the second set revealed many disputable points and it become
apparent that these patches will need further changes.
The second part of my changes is focused mostly on conflicts detection and allows for almost all checks to be removed from the code as ad-hoc solutions, as the new structures and functions take care of them automatically. Any programmer making a change just has to correctly specify values in a {\tt struct opt\_params}, write in a list of conflicting options, and the validation of the new option is guaranteed to work correctly and seamlessly.

Size of this patchset in the first RFC is 22 patches and the total changes are:
\begin{lstlisting}[frame=none, basicstyle=\footnotesize\ttfamily, language=Bash, numbers=none, numberstyle=\tiny\color{black},caption= {Git statistics for the second patchset~\cite{secondSetRFC}.}]
 mkfs/xfs_mkfs.c | 2952 +++++++++++++++++++++++++++++++++++--------------------
 1 file changed, 1864 insertions(+), 1088 deletions(-)
\end{lstlisting}


